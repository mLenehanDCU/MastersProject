\IEEEPARstart{I}{nternet} of Things devices have seen a large increase in usage
in the past number of years. With estimates of the number of devices exceeding
40 billion\cite{idc_2019}, and the amount of data produced by these devices in the order
of Zettabytes\cite{idc_2019} ($10^{21}$), it is apparent that the recent issues
regarding Distributed Denial of Service (DDoS) attacks may be concerning to
those utilising the networks.

The motivation of this project is to evaluate the susceptibility of IoT
operating systems and devices to these types of attacks. The goal is to develop
or recommend a solution for the detection of an attack, and the mitigation of
such an attack.

\subsection{Distributed Denial of Service Attacks}

Distributed Denial of Service attacks aim to shut down their targets servers
through a high volume flood of traffic. This high volume traffic, which can be
in the order of Terrabytes of data, serves to overload the available resources
of the hosts network, which causes regular user traffic to be rerouted. This
re-routing leads to the ``denial of service'' aspect of the attack.

In order to generate such large volumes of data, an attacker uses multiple
devices for transmission of malicious data packets. The ``distributed'' aspect of
the DDoS attack comes from the multiple devices which are infected with
malware, placed on the devices by the attacker. These devices are known as bots, and
are remotely accessed by an attacker to generate high traffic to drain the
networks resources.

\subsection{Internet of Things DDoS Attacks}

A DDoS attack relies on the attacker gaining remote access to devices in order
to initiate a flood of traffic. Due to their relative abundance, Internet of
Things devices have become a target for these types of attacks in recent years.
IoT devices can act as an entry point to the network for an attacker due to the
``always-on'' nature of the devices and their connection to the network, and
their lack of security features.

Attacks, such as the Mirai attack, rely on the default security credentials,
which tend to remain unchanged when configured by end users, to gain access to
the devices for the uploading of malicious software. Once this
malware is uploaded to the device, the attacker can begin to flood the desired
network with traffic\cite{7971869}.

Since the Mirai attack in 2016, and the subsequent release of the Mirai source
code, there has been an increase in the number of IoT related DDoS
attacks\cite{Salim2019}. As such, it is clear that there is a need to secure
these devices, and the networks in which these devices are utilised. As IoT
devices become more prevalent in different areas and industries, it becomes
increasingly important to not only protect the data being generated, but also to
protect the greater network as a whole which is processing this data.

\subsection{IoT Areas}

Internet of Things devices have become common place in everyday life, with
devices such as internet connected security cameras, smart-home devices, and
wearable smart devices being common in households. However, there are many other
use cases outside of the home where IoT devices have been adopted.

Industrial Internet of Things (IIoT) includes devices such as cameras and sensors
within industrial settings, which, as with IoT, provides the ability to offload
intensive processing to higher performance servers\cite{iiot_2019}. This division
of IoT devices is integral for the ``Industry 4.0'' concept. Devices within this
sector can be in control of, for example, safety systems, and as such, their
continuous operation is of vital importance.

Internet of Medical Things (IoMT) devices include healthcare specific connected
devices, such as medical sensors and wearable devices for patient monitoring
purposes\cite{iomt_2019}. These devices allow for an increase in patient comfort,
and are an integral component of the ``Healthcare v4.0'' concept. With an
increase in the number of attacks on similar systems, there is the concern that
the implementation of such systems could be hindered if more robust security
systems are put in place.
