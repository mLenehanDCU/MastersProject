With the increasing number of IoT devices being used within industry, and the
growing importance of IIoT within the ``Industry 4.0'' concept, there is an
increasing need to secure the data and data systems being used. Within
industrial settings, the data being handled by these IoT systems could be vital
in terms of safety systems, or mission-critical systems within production. As
such, Yan et al. have presented a Multi-level DDoS mitigation
framework (MLDMF) which utilises software defined networking (SDN) for the
purposes of managing and securing IoT devices\cite{iiot_2019}.

Yan et al. acknowledge that the defence systems available to IoT devices are
``unsubstantial''. The MLDMF proposed consists of three levels, the edge, fog,
and cloud computing levels. Each of these levels serves its own purpose in
securing the IIoT network. The edge computing level provides gateways with
higher computational power than the IoT devices used in the network, which can
implement more ``traditional'' security features, such as IP address
concealment, and malicious software detection.

The fog computing level analyses data from the edge computing level, and, using
a number of techniques, can make decisions on how to handle the data. The three
described techniques are collect-detect-mitigate (CDM), honey-pot-detect-react
(HDR), and cloud-detect-fog-mitigate (CDFM). CDM allows for modifications to the
networks bandwidth based on properties of the traffic. HDR can redirect traffic
to a virtual IoT device in order to perform analysis. This technique allows for
stopping malware from affecting the network, and allows for modifications to
policies based of any new information gained from captured traffic. CDFM shares
information between the cloud and fog computing levels in order to inform
decision on dropping packets which may be from malicious sources.

The cloud computing level implements machine learning techniques in order to
analyse the large amounts of data flowing through the network. This information
can be used for the detection and prediction of DDoS attacks, based on the
incoming data. Experimentally, the implementation of the proposed MLMDF improves
performance within a network undergoing a TCP SYN flood attack by approximately
37.03\%.

The use of SDN's for management of IoT devices and the traffic within the
network have the disadvantage of adding overhead to the network. There is also
the added cost of hardware for the implementation of the proposed system,
requiring honeypot servers, and gateways to act as controllers. While the
proposed solution may be viable, and these costs may be minimal in the context
of an industrial setting, the current scope of the project is more focused on
implementing security at the IoT device level.
