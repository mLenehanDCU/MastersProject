There is a clear need for a lightweight algorithm for the encoding and decoding
of information being transmit between devices in an IoT network. Noura et al.
present such an algorithm in the context of Multimedia IoT (MIoT) devices. The
proposed solution uses pseudo-random keys for encryption, greatly increasing the
difficulty of decoding intercepted transmissions\cite{iotalgorithm}.

A number of constraints were considered in the development of this algorithm,
including the limited resources available to IoT devices, and the need for an
algorithm which can be implemented across devices of varying performance. This
leads to a lightweight algorithm which may be used across a number of different
devices, regardless of the resources available to that device.

In order to test the robustness of this algorithm, a number of tests were
performed, separated under headings which include ``Statistical analysis'', ``Visual
degradation'', and ``Resistance against well-known types of attacks''. With
regards to the statistical analysis, the algorithm was shown experimentally to
have very low correlation between adjacent pixels of the encoded image, with
output values of close to zero. With regards to visual degradation, there was a
signal-to-noise ratio (SNR) of 8.5894dB, which is regarded within the study to
be a ``low value'', showing that the algorithm has acceptable robustness to
degradation during encoding and decoding. Through a number of attack tests, the
algorithm was shown to be robust to statistical attacks, and the size of keys
used attributed to robustness to brute force attacks.

In terms of performance, when tested on two common embedded Linux platforms, the
Raspberry Pi Zero W, and the Raspberry Pi 2, this algorithm gave encryption time
gains of up to 29\%, and decryption time gains of up to 33\% when compared with
a similar implementation.

One suggestion made in this study is that the algorithm has been implemented in
C, whereas the implementation compared to with regards to performance is
implemented in assembly. As such, further performance benefits could be gained
from this lower level implementation.

This algorithm proposal demonstrates that a lightweight, ``flexible''
cryptographic algorithm can be implemented in a robust and highly performing manner.
While the proposed algorithm is intended to be used in the context of multimedia
transmissions, and is tested in the transmission of an image, it is clear that
the uses of the algorithm could be extended beyond this scope.
