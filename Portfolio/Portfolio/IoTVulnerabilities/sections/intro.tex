There are an estimated 31 billion Internet of Things devices currently in
use\cite{jan13}.
As these devices are being adopted in all aspects of life - from home devices,
medical devices, industrial devices - there are valid concerns raised about the
security capabilities of these devices. The requirement for improvements to the
resilience of these devices to attack is clear following a number of large scale
botnet attacks, including the Mirai attack.

The Mirai attack was able to generate a massive 1Tbps of data, and targeted the
Dyn DNS service provider. It is estimated that over 600,000 devices were
a part of the Mirai botnet. The Mirai code exploits poor user credential
implementations on a variety of consumer IoT devices. Mirai uses 60 hard-coded
username and password pairs to gain access to the devices. These credentials are
a combination between factory-set, and common user-set username and password
pairs.

The ability for a botnet to grow this large, and to launch such a large scale
flooding attack leads to the questioning of the security implementations on any
connected embedded system, and of the other vulnerabilities which may be
exploited.

The aim of this evaluation is to identify the vulnerable areas of IoT security,
whether this is user authentication, transport protocol, or encryption based.
