Neshenko et al.\cite{neshenko_2019} define 9 classes of IoT vulnerabilities. These are
deficient physical security, insufficient energy harvesting, inadequate
authentication, improper encryption, unnecessary open ports, insufficient access
control, improper patch management capabilities, weak programming practices, and
insufficient audit mechanisms. While a number of these are not relevant to the
susceptibility of these devices to malicious attack for their future use in a
large scale botnet, for example physical security - an attack is not going to
physically connect their PC to thousands of IoT devices to build their botnet -
issues such as improper and encryption authentication methods, open ports, and
an inability to update flaws in firmware certainly leave these devices
vulnerable to attack on a large scale.

\subsubsection{Common Vulnerabilities and Exposures - IoT Logs}

CVE, or Common Vulnerabilities and Exposures, is a publicly accessible list of
cybersecurity vulnerabilities\cite{cveHome}. This list is comprised of reported
vulnerabilities from user and engineers, with details of the vulnerability and
level of exposure to the system due to the vulnerability. Each CVE entry
contains a CVE ID, details of the vulnerability, and Common Vulnerability
Scoring System (CVSS) score. The CVSS score is calculated using a number of
metrics, including at its base level the attack vector and complexity, the
requirement for user interaction, and the impact of the vulnerability on a users
confidentiality or the availability of the service\cite{nvdCalculator}. The
score defines a severity rating from Low to High for the listed vulnerability.
With the CVSS 3.0 rating system, scores from 0.1 to 3.9 are low severity, 4.0 to
6.9 is medium severity, 7.0 to 8.9 are high severity, and 9.0 to 10 are critical
severity.

Listed in Table \ref{tab:iotCve} are the currently listed CVE reports for the IoT
operating systems RIOT, Contiki, Amazon FreeRTOS and Zephyr OS.

\begin{table}[H]
	\centering
	\caption{IoT Operating System CVE Logs}
	\label{tab:iotCve}
	\begin{tabular}{|c|c|c|c|}
	\hline
	Operating System & CVE ID & CVE Score \\
	\hline\hline
	RIOT-OS & CVE-2019-16754 & 5.0 \\
		& CVE-2019-15702 & 5.0 \\
		& CVE-2019-15134 & 7.8 \\
	\hline
	Contiki-OS & CVE-2017-7296 & 4.3 \\
		   & CVE-2017-7295 & 7.8 \\
	\hline
	Amazon FreeRTOS & CVE-2018-16603 & 4.3 \\
			& CVE-2018-16602 & 4.3 \\
			& CVE-2018-16601 & 6.8 \\
			& CVE-2018-16600 & 4.3 \\
			& CVE-2018-16599 & 4.3 \\
			& CVE-2018-16598 & 4.3 \\
			& CVE-2018-16527 & 4.3 \\
			& CVE-2018-16526 & 6.8 \\
			& CVE-2018-16525 & 6.8 \\
			& CVE-2018-16524 & 4.3 \\
			& CVE-2018-16523 & 5.8 \\
	\hline
	ZephyrOS & CVE-2018-1000800 & 7.5 \\
		 & CVE-2017-14202   & 4.6 \\
		 & CVE-2017-14201   & 4.6 \\
		 & CVE-2017-14199   & 7.5 \\
	\hline\hline
	\end{tabular}
\end{table}

\subsubsubsection{\textbf{RIOT OS Vulnerabilities}}

The listed RIOT OS vulnerabilities range from medium to high severity. The two
medium severity vulnerabilities are protocol vulnerabilities. The first of these
is an issue with the OS's implementation of the MQTT messaging
protocol\cite{riotCve1}. Due to
the implementation, null pointer dereferencing results in a segmentation fault,
creashing the system\cite{riotMqtt}. The second of these protocol vulnerabilities
is a TCP parsing error\cite{riotCve2}, whereby the system can be places in an infinite loop while
attempting to parse a packet with an  unknown option specified, and an option
length of zero\cite{riotTcp}.

The highest severity vulnerability listed for RIOT OS is another protocol
vulnerability. This is cited as exposing the devices running this operating
system to Denial of Service attacks. An error with the implementation of the TCP
handshake, in which a memory leak occurs when an ACK is received by the device
as the first packet in the handshake instead of a SYN packet\cite{riotCve3}. This memory leak
can lead to an exhaustion of memory resources on the device.

\subsubsubsection{\textbf{Contiki-OS}}

Both the first and second  listed CVE reports for Contiki-OS refer to the same
Cross-Site Scripting vulnerability in the mqtt.html configuration page. The
first of these reports is a medium severity vulnerability\cite{contikiCve1},
while the second is listed as a high severity vulnerability\cite{contikiCve2}.
This difference in severity is due to the level of detail given in the CVE
reports. This vulnerability in the operating systems included MQTT configuration
implementation causes a null pointer dereference error, which can be initiated
via a HTTP POST request, and can cause the host device to crash, causing a denial of
service\cite{contikiMqtt}.

\subsubsubsection{\textbf{Amazon FreeRTOS}}

\subsubsubsection{\textbf{ZephyrOS}}
