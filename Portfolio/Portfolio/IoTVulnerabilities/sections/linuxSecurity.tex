Non-resource constrained, "fully fledged" operating systems which run on
consumer PCs and enterprise server hardware have a number of security features
which can be implemented to protect against attack. These range from firewall
and packet filtering software, to firmware verification software.

\subsection{Secure Boot}

Secure boot is a security mechanism used to verify that software being run on a
system is from a trusted source. The vendors whose software has been granted
permission to run on the system are stored in firmware\cite{ubuntuSecureBoot}.
This allows for any software being loaded onto the device at boot time can be
verified by comparing the vendor signature against the keys which are stored in
firmware.

\subsubsection{UEFI Secure Boot}

UEFI found the need to implement the secure boot mechanism as protections for
post-boot code injection had become competent enough that attackers were
beginning to focus their efforts on pre-boot firmware injection. Pre-boot
malware injection includes rootkit and bootkits, which, before secure boot,
proved extremely difficult to detect.

There are a number of intended results for pre-boot exploits, ranging from
control over the operating system, or user identification credential snooping,
to control over ROM and PCIe peripherals.

Vendor signatures are stored in databases in the devices firmware. The two types
of keys used for accessing these databases are Platform Keys, and Key Exchange
Keys\cite{uefiSecureBoot}. Platform keys are the vendor defined key, which
prevents modifications to the Key Exchange Key. The Key Exchange Key
prevents modifications to the signature database. This mechanism allows vendors
with valid KEK's to modify the signature database, such as inserting or deleting
a signature. By comparing each code module's signature to the database at boot
time, a decision can be made on whether to load this module and proceed with the
boot process, or not to load the module and terminate the boot process.

The device manufacturer controls the signatures which are included from the
factory. Vendors such as Microsoft offer programs for vendors to submit modules
for authorization, to have their signatures added to the signatures database.
This system has been mirrored by a number of Linux distributions.

Secure boot is compatible with the x64 platform, and is enabled by default on
Windows and most Linux distributions\cite{hoffman_2017}. By enabling this
mechanism by default, the user must disable the feature to open themselves to
vulnerability.

\subsubsection{Secure Boot in IoT Systems}

While IoT devices typically have a one-time setup, they can be rebooted multiple
times over there lifetime, especially in the case of smart home IoT devices, as
they are relocated or power cycled. By initiating their boot process, these
devices become vulnerable to the injection of malicious code modules. There is a
recognition for the need of a secure boot type mechanism for Internet of Things
devices, with implementations such as MCUboot in development for a number of
microcontroller based operating systems such as RIOT and
Zephyr\cite{juullabs-oss}. There are, however, a number of limitations which
have to be considered, namely the lack of available storage on these contrained
devices. As a workaround for this constraint the boot code can be written to
non-modifiable ROM, although this could raise issues in the devices future, as
patches required to the devices firmware to mitigate other vulnerabilities could
not be implemented\cite{iotSecurityFoundation}.

\subsection{IPtables}
