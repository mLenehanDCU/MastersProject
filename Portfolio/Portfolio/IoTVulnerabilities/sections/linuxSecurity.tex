Non-resource constrained, "fully fledged" operating systems which run on
consumer PCs and enterprise server hardware have a number of security features
which can be implemented to protect against attack. These range from firewall
and packet filtering software, to firmware verification software.

\subsection{UEFI Secure Boot}

Secure boot is a security mechanism used to verify that software being run on a
system is from a trusted source. The vendors whose software has been granted
permission to run on the system are stored in firmware. This allows for any
software being loaded onto the device at boot time can be verified by comparing
the vendor signature agains the keys which are stored in firmware.

Secure boot is compatible with the x64 platform, and is enabled by defautl on
Windows and most Linux distributions.

\subsection{IPtables}
