With botnets being used to launch large scale DDoS attacks, techniques must be
implemented to detect these large scale attacks as they begin to unfold, and to
mitigate the impact on a service due to these attacks. With denial of service
attacks, the volume of traffic which could be generated by one attacking source
was much less than that of a distributed botnet. As attack traffic was coming
from a single source, detection was a much easier feat. As noted by Furfaro et
al.\cite{furfaro}, the large number of hosts make identifying the attack source
extemely difficult.

\subsection{Detection}

Meidan et al.\cite{meidan} presents a method of detecting vulnerable IoT devices on a
home network by training a machine-learning classifier to identify devices by
model, and if any known vulnerabilities exist either attempting to update the
devices firmware or rerouting traffic from the device. This type of detection
technique detects device vulnerabilities, rather than the occurrence of a denial
of service attack, however, it is a useful implementation as it can allow for
the removal of the opportunity for a device to be exploited and used in a DDoS
attack.

\subsection{Mitigation}

