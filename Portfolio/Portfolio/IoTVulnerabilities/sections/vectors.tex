There are a number of ways in which an attacker can gain access to, or completely
shut down a device. These are known as attack vectors. Attack vectors in terms
of the Internet of things can be vulnerabilities due to the encryption used, the
transport protocol, or the user authentication. Mohamad Noor et al. breaks the
IoT network architecture into three layers, the Application Layer, the Network
Layer, and the Perception Layer\cite{mohamadSec}. The perception layer deals
with the device hardware. The network layer deals with the protocols used for
device communications. Finally, the application layer deals with the services
connected to the devices, such as a cloud computing platform. Each layer can be
linked to one of the aforementioned attack vectors.

\subsection{User Authentication}

User authentication methods within IoT, as shown by the Mirai botnets, is an
area of clear exposure for devices. Due to the overhead which is associated with
implementing complex authentication systems, such as the LKA
protocol\cite{mohamadSec}, the devices manufacturer may choose to forgo these
implementations in order to reserve resources for the devices critical
functionality.

At the application layer, Ammar et. al present how Amazon Web Services' (AWS)
IoT framework provides authentication using the Identity and Acess Management
service of the AWS cloud computing platform\cite{ammarSec}. AWS IAM is a policy
based authentication service. Using this service, any IoT device connected to
the AWS IoT cloud is given access to resources within the cloud based on a
user-defined access policy.

\subsection{Encryption}

Encryption tends to suffer from the same issues as user authentication
protocols, in that manufacturers often choose to forgo a more complex
implementation to reserve devices resources.

