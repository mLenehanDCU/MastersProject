The focus of this project will be to evaluate IoT systems in terms of the
devices, networks, and operating systems for their susceptibility to DDoS
attacks. As such, the survey of Salim et al.\cite{Salim2019} provides an
excellent background in the types of attacks, and is a starting point for
assessing the types of solutions which are available to deal with these attacks.
By understanding the types of attacks which commonly occur, more accurate
evaluations can be performed between IoT systems.

IIoT is becoming a large source of data, and accounts for a large number of IoT
devices. The proposal laid out by Yan et al.\cite{iiot_2019} gives an insight
into how security issues within industry are being approached. This type of
system can be investigated as an IoT network as a whole. Simulation of this type
of network is a possible solution for the evaluation of a network consisting of
SDN gateways and multiple controllers.

The suggestions laid out in the work of Yaacoub et al.\cite{iomt_2019} can
clearly be applied to IoT networks and devices of all types, and are not
specific to the field of IoMT. The implementation of a lightweight
authentication protocol for IoT networks is an area which could be investigated
further within the scope of the project.

The dynamic structure cryptographic algorithm proposed by Noura et
al.\cite{iotalgorithm} shows promising results in terms of relative performance
and robustness. An evaluation of these algorithms should be further investigated,
however an implementation would be considered as outside of the scope of this
project. The proposed algorithm appears to solely be implemented
for the purpose of transmitting images, however the extension of this algorithm
to other data types could prove useful for security.

It should be noted that these solutions can all potentially be implemented in
various combinations. An in depth evaluation should consider any permutation of
the aforementioned solutions. For the purposes of testing and evaluating these
solutions, either simulation or virtualization will be used. While simulation
will require less available computing resources, it is possible that
virtualization of the network, as described in \cite{iiot_2019} will provide a
more complete insight to the effects of changes to network level, or operating
system/device level parameters.
