The medical applications of IoT devices have become apparent in recent years.
However, with the sensitive nature of the data being generated, privacy and
security are of major concern. Yaacoub et al. present an in depth analysis of
the dangers of IoMT devices, a number of commonly faced attacks, and a
discussion on the possible solutions to these issues\cite{iomt_2019}.

The three relevant recommendations made within this study are the implementation
of lightweight cryptographic algorithms for security, alongside a lightweight
authentication protocol, and a layered IoT secutrity architecture. The lightweight
aspect is key in this assessment, as IoT devices do not have the resources to
implement more robust solutions. One suggested security algorithm is the dynamic
structure cryptographic algorithm, which is discussed further in Section
\ref{sec:iotAlg}.

The layered security architecture, much like the MLDMF, separates security
responsibilities between different levels. Unlike the MLDMF however, these
layers define Quality of Service (QoS) parameters. Due to the nature of the
application, IoMT devices must give accurate, real-time feedback. As such, this
``zero-tolerance'' for error must be represented within the QoS demands of the
system.

The proposed use of dynamic structure cryptographic algorithms, and lightweight
authentication protocols would greatly improve security within IoT networks. A
low latency, non-resource intensive authentication scheme would allow for
transmission of data between trusted devices, while allowing real-time
processing of data. While focused mainly on IoMT, the insight provided by this
study can be translated to IoT in general.
