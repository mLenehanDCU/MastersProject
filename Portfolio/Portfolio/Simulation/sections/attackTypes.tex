For testing purposes, a number of different attack types were be simulated. SYN
Flood attacks and ICMP Flood attacks were be simulated.

\subsection{SYN Flood}

SYN floods are an attack which exploit the TCP protocols TCP handshake. The
attacking node sends TCP SYN packets to the target using spoofed IP addresses.
The target device will reply with a SYN-ACK, and wait for the initiator to reply
with a final ACK packet. As the attacking node is using a spoofed IP address, it
never responds to the SYN-ACK, and so the target node will wait indefinitely for
the ACK packet\cite{cfSynFlood}.

\begin{figure}[H]
	\centering
	\includegraphics[width=0.8\textwidth]{images/SYNAttack}
	\caption{SYN Flood Attack}
	\label{fig:images-SYNAttack}
\end{figure}

In order to perform a SYN Flood attack, the "hping" tool is used. Hping is a
tool for creating and transmitting TCP, UDP or ICMP packets. This tool has
multiple functions, such as port scanning, and firewall mapping, but can also be
utilised as a Denial of Service tool. By utilising the tools "rand-source" and
"flood" flags, the tool can execute a SYN flood, sending TCP SYN packets to the
target address via spoofed IP addresses.

\subsection{ICMP Flood}

An ICMP flood is a distributed denial of service attack, performed using a
botnet\cite{cfIcmpFlood}. ICMP requests take server resources to process and reply to. By sending
large numbers of ICMP requests, the servers resources can be exhausted.

\begin{figure}[H]
	\centering
	\includegraphics[width=0.8\textwidth]{images/ICMPFlood}
	\caption{ICMP Flood Attack}
	\label{fig:images-ICMPFlood}
\end{figure}

The "ping" tool can be used to execute an ICMP flood. By using the "-f" flag,
the ping tool can be used to launch an ICMP flood from a single attacking node.
