For work built upon these simulations, the first recommendation is to use either
the Mininet VM image, or a Linux distribution such as Ubuntu. While Manjaro is
based on Arch Linux, which has access to a large number of packages, Mininet,
along with a number of the packages used for sending and monitoring traffic flow
were required to be installed from the Arch User Repositories (AUR). For
comparison, Mininet, is available from Ubuntus default "apt" package manager,
along with all of the other software packages used.

While there was a mitigation technique implemented in the Denial of Service
simulations, there was no mitigation technique implemented in the Distributed
Denial of Service simulations. This was due to the relative complexity of
implementing a technique which could deal with such a large number of hosts. For
future work building off these simulations, a mitigation technique could be
implemented in order to test it's effectiveness in lowering the impact
experienced during a large scale security event such as a DDoS attack.

For the Distributed Denial of Service attack simulations, a total of 64 host
nodes were implemented, with 28 of these used for the purposes of executing the
attack. It is reported on the Mininet overview page that up to 4096 hosts have
been successfully implemented on a single machine, however there is no
indication given for the level of performance achieved with this number of
running hosts. During testing there was a significant amount of lag experienced
on the host PC's operating system while only 28 nodes were running. With a more
powerful host machine, future work could implement simulations with much larger
numbers of attacking nodes.
