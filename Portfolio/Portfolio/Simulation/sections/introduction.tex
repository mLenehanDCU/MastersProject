The aim of a Distributed Denial of Service attack is to disrupt the connection
between a service, such as an application server or DNS server, and that
services users.

Denial of Service (DoS) attacks typically exploited vulnerabilities with network
or application layer protocols, for example, Syn Floods or HTTP Floods. With these
types of attacks, spoofed IP addresses were typically used, both to mask the IP
address of the attacker, and to take advantage of vulnerabilities of the
protocols being used.

As DoS mitigation techniques improved, and attacks using IP spoofing became
easier to defend against, attackers began utilising botnets to perform
amplification attacks. Distributed Denial of Service (DDoS) attacks utilise
large number of hosts to perform attacks. These types of attacks prove more
difficult to defend against than typical Denial of Service attacks.

In order to extensively evaluate the impact of Distributed Denial of Service
attacks, and the role which IoT devices play in modern DDoS attacks, simulations
can be used. Simulations can show the impact to a network, and the disruption
caused by these attacks, without having to use real network devices.

\subsection{Simulation Software}

There were a number of simulation software packages considered for the tests
contained in this document. Each of these packages offer their own set of
advantages and disadvantages.

\subsubsection{ns-3}

The most commonly used network simulation software is "ns-3"; a "discrete-event
network simulator for Internet systems"\cite{nsnam_20AD}. This package is
commonly used for research purposes, and numerous examples of D/DoS simulations
implemented using ns-3 can be found\cite{}\cite{}.

\subsubsection{Cooja}

Another simulation software package considered was "Cooja". This package
simulates Wireless Sensor Networks built on the Contiki IoT operating
system\cite{cooja_2019}. While the underlying system being simulated is an IoT device,
there was little information to be found on implementing the types of testing
required, and as such this software package was ruled out.

\subsubsection{Mininet}

The final software package considered was "Mininet". Mininet is a network
simulator which uses Linux containers to emulate network
devices\cite{mininet_2018}. As the network devices are emulated using containers,
each host uses real Linux Kernel code.

As the Mininet hosts run Linux Kernel code, Linux applications can be easily
run. This is advantageous as common attack tools can be installed and executed.
Each host can be assigned a unique IP address allowing the attacking host to
address it's target as in a real attack situation. As each host is implemented
as a container, a terminal can be open for each host for manual command execution.

Mininet implements Software Defined Networking using the OpenFlow protocol. By
default, Mininet uses the OpenFlow reference controller, however, it also allows
for the use of remote controllers. The "Floodlight" controller will be used for
this purpose, as it has a GUI application which can display the network
topology.

Mininet virtualizes the network links between each host or switch in the
network. These links can be assigned parameters such as a delay time on the
link, or a set bandwidth.

The Python API implemented by Mininet allows for the topology and the terminal
commands to be created and executed programmatically. The Python API will be
used in order to execute repeatable tests with reproducible results.

Due to the advantages offered by the Mininet simulation software package, this
will be the simulator of choice for testing. There are however a number of
disadvantages which must be noted\cite{mnov_2018}.

\begin{enumerate}
	\item Containers share the file system of the host, i.e. the desktop or
		VM on which Mininet is being run
	\item A network cannot exceed the bandwidth of a single server
	\item Non-Linux-compatible OpenFlow switches and applications are not
		supported
\end{enumerate}

While these limitations must be kept in mind, they will not prove to be a
hindrance for the purposes of these simulations.
