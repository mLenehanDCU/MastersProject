The following steps must be performed in order to reproduce the achieved
simulations. As the Manjaro Linux distribution was used for these simulations,
all package manager instructions will be given for Manjaro.

\subsection{Mininet}

Mininet must first be installed. This can be done according to the Mininet
documentation \ref{}. For Debian based systems, Mininet can be installed via the
"apt" package manager, using the command:

\begin{lstlisting}[language=bash, caption=Debian-based Distro Mininet Install]
$ apt-get install mininet
\end{lstlisting}

For Arch based distributions, Mininet may be installed using the Arch User
Repository using the following command:

\begin{lstlisting}[language=bash, caption=Arch-based Distro Mininet Install]
$ yay -S mininet-git
\end{lstlisting}

To test the installation, execute the command:

\begin{lstlisting}[language=bash, caption=Mininet installation test]
$ sudo mn --test pingall
\end{lstlisting}

If the error "ovs-vsctl: unix:/run/openvswitch/db.sock database connection
failed" occurs, the Open vSwitch must be started using the command:

\begin{lstlisting}[language=bash, caption=Open vSwitch service start command]
$ sudo /usr/share/openvswitch/scripts/ovs-ctl start
\end{lstlisting}

\subsection{Floodlight OpenFlow Controller}

The Floodlight OpenFlow Controller can be installed via GitHub\ref{}. For
version 1.2 (as used for these simulation), the Java 8 development kit must be
installed:

\begin{lstlisting}[language=bash, caption=openjdk8 installation]
$ sudo pacman -S openjdk8-src
\end{lstlisting}

Floodlight has a number of dependencies which can be installed via the following
command:

\begin{lstlisting}[language=bash, caption=Floodlight Dependencies]
$ sudo pacman -S git ant maven python-dev
\end{lstlisting}

To download and build Floodlight, execute the following commands:

\begin{lstlisting}[language=bash, caption=Floodlight installation commands]
$ git clone git://github.com/floodlight/floodlight.git
$ cd floodlight
$ git submodule init
$ git submodule update
$ ant
# If the ant build fails, Floodlight can be built using the following Maven command
$ sudo mvn package
\end{lstlisting}

\subsection{Source Code}

The source code for the simulations can be found on GitHub, and can be
downloaded using the following command:

\begin{lstlisting}[language=bash, caption=Source code download]
$ git clone https://github.com/mLenehanDCU/MininetCode.git
\end{lstlisting}
